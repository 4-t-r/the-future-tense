\section{Discussion}

\subsection{Topic Assignment}
As described in section \ref{results} the most statements are assigned with the "machine human interface" topic.
This is not surprising, since technical sources clearly dominate our websites. 
Technological systems often require human operation and contain a corresponding interface as an implicit feature.
Also search engines often need the handling from the outside. 
For this reason, many of these search engine related phrases end up in this category instead of in the search engine topic.
Since search engines also correspond to natural language systems, they are often assigned to the category "natural language technology".
To avoid this, more distinct generic terms for topics should have been chosen.
This issue also arises in the field of "human machine interface". 
Statements that describe a friendly AI are grouped in this category . 
We would expect this kind of sentences in "transhumanism".
The reason for this could be the name, since it combines the terms human and machine.
Furthermore, some of the future statements deal with two of the topics we have defined at the same time. 
For instance, some sentences refer to both social media and search engines. However, these sentences can only be associated with one of the two categories.
With the help of fuzzy clustering, phrases could also be mapped to multiple subjects.
\\
"Computer vision", "natural language technology", "social media", "transhumanism" and "human machine interface" contain many statements that do not fit the corresponding topic.
Whereas this appears quite random for "natural language technology" and "human machine interface", since we did not define suitable topics for these sentences.
In "computer vision" many sentences end up containing scan, look, see.
But these do not actually refer to the topic.
Similarly "social media" includes  statements with the word fans, even if this sentence is about attending soccer match.
But also in this category are plenty of randomly assigned statements.
Nevertheless, the categories "computer vision" and "natural language technology" also contain appropriate phrases.  
The quality of these subjects is mixed.
"human machine interface" consists of far too many terms that do not fit to this topic. 
A possible solution for such random assignments could be to add the generic category "others".
Furthermore, it should be considered whether the category human machine interface is actually useful.
It does not seem specific enough to narrow down a particular topic.
\\
The Topic Model seems to assign the "research computing" category to statements more reliably.
Nevertheless, it is noticeable that educational topics that cannot be associated with research and hardware or software are also grouped under this theme.
Sentences regarding technical topics, which have no connection with research, also often fall into this category.
However, in the end it is a matter of interpretation whether these statements can really not fall under "research computing".
\\
Finally, there are also topics that seem to be correctly assigned to their statements for the majority of the time.
The corresponding categories are finance and gaming.
This could be an indication that the model is not suitable for too specific technical terms.
Providing more common topics contained in the everyday english language use, could result in a more reliable topic distribution.

\subsection{Website Examination}

The final data set, produced by the model pipeline, contains the url for every AI future statement.
Table~\ref{Top-Domains} presents the domains that are mostly occurring in this final data set.
When examining the main domains, these appear relatively  diversified.
A website dealing with phylosophical questions on the topic of AI is included (lesswrong.com).
Following this, there are three sites from the field of gaming (acceleratingfuture.com, mugenguild.com, slightlymagic.net).
Also, the blog of the department of defense is contained among these domains dealing with the research of defence and military needs (dodsbir.net).
A store with speech recognition devices is also available (knowbrainer.com).
Nevertheless a number of scientific blogs on AI-related topics are also include, which are lead by researchers or from the tech industry.
Latter are mostly data scientists.
Considering the other domains, many scientific websites as well as websites about gaming are also very abundant.
Thus, rather the future statements were expressed by people from AI related fields.
This could mean that this topic has a lower role in the general population and thus it is dealt very little with AI-specific topics in the public society.
New discoveries could be made by observing domains containing statements from the last few months were used.
More people might feel affected by the latest developments in this area.
Consequently, there could be more blogs with people from other sectors who would exchange opinions about these developments.
%%%%%%%%%%%%%%%%%%%%%%%%%%%%%%%%%%%%%%%%%%%%%%%%%%%%%%%%%%%%%%%%%%%%%%%%%%%%%%%%%%%%%%%%%%%%%%%%%%%%%%%%%%%%%%%%%%%%%%%%%%%%%%%%%%%%%%%%%%%%%%%%%%%%%%
%%%%%%%%%%%%%%%%%%%%%%%%%%%%%%%%%%%%%%%%%%%%%%%%%%%%%%%%%%%%%%%%%%%%%%%%%%%%%%%%%%%%%%%%%%%%%%%%%%%%%%%%%%%%%%%%%%%%%%%%%%%%%%%%%%%%%%%%%%%%%%%%%%%%%%
\begin{table}
    \setlength\tabcolsep{2pt} % let LaTeX compute intercolumn whitespace
    \centering
    \captionsetup{size=footnotesize}
    \resizebox{\columnwidth}{!}{%
    %\begin{tabular*}{\columnwidth}{@{\extracolsep{\fill}}%
    %%%%%%%%%%%%%%%%%%%%%%%%%%%%%%%%%%%%%%%%%%%%%%%%%%%%%%%%%%%%%%%%%%%%%%%%%%%%%%%%%%%%%%%%%%%%%%%%%%%%%%%%%%%%%%%%%%%%%%%%%%%%%%%%%%%%%%%%%%%%%%%%%%%%%%
    \begin{tabular}{rll}
        \toprule
        \textbf{AI statements} & \textbf{Website} & \textbf{Description} \\
        \midrule
        \multirow{2}{*}{210\quad} & \multirow{2}{*}{lesswrong.com}          & Philosophical blog \\
                                  &                                         & about AI developments \\
        \addlinespace[0.7em]
        \multirow{2}{*}{198\quad} & \multirow{2}{*}{arcengames.com}         & Page of an indie \\
                                  &                                         & game developer \\
        \addlinespace[0.7em]
        \multirow{2}{*}{182\quad} & \multirow{2}{*}{acceleratingfuture.com} & Blog about perspectives \\
                                  &                                         & and emerging technologies \\
        \addlinespace[0.7em]
        \multirow{2}{*}{156\quad} & \multirow{2}{*}{heatonresearch.com}     & Blog of a data \\
                                  &                                         & scientist \\
        \addlinespace[0.7em]
        \multirow{2}{*}{106\quad} & \multirow{2}{*}{dodsbir.net}            & Research blog of the \\
                                  &                                         & department of defense \\
        \addlinespace[0.7em]
        \multirow{2}{*}{76\quad}  & \multirow{2}{*}{kdnuggets.com}          & Blog of data scientists for \\
                                  &                                         & analytics and machine learning \\
        \addlinespace[0.7em]
        \multirow{2}{*}{71\quad}  & \multirow{2}{*}{knowbrainer.com}        & Shop containing speech \\
                                  &                                         & recognition devices\\
        \addlinespace[0.7em]
        {58\quad}                 & mugenguild.com                          & 2D fighting game \\
        \addlinespace[0.7em]
        {52\quad}                 & aidreams.co.uk                          & Robotics and AI blog\\
        \addlinespace[0.7em]
        \multirow{2}{*}{51\quad}  & \multirow{2}{*}{slightlymagic.net}      & Rules Engine for the game \\
                                  &                                         & ``Magic: the Gathering'' \\
        \bottomrule
    \end{tabular}
    }
\caption{\label{Top-Domains}
Top Domains
}
\end{table}
%%%%%%%%%%%%%%%%%%%%%%%%%%%%%%%%%%%%%%%%%%%%%%%%%%%%%%%%%%%%%%%%%%%%%%%%%%%%%%%%%%%%%%%%%%%%%%%%%%%%%%%%%%%%%%%%%%%%%%%%%%%%%%%%%%%%%%%%%%%%%%%%%%%%%%
%%%%%%%%%%%%%%%%%%%%%%%%%%%%%%%%%%%%%%%%%%%%%%%%%%%%%%%%%%%%%%%%%%%%%%%%%%%%%%%%%%%%%%%%%%%%%%%%%%%%%%%%%%%%%%%%%%%%%%%%%%%%%%%%%%%%%%%%%%%%%%%%%%%%%%
\subsection{Project Limitations}

Since we had a limited time for this project, there are some aspects where we would have liked to continue our work.
From a technical point of view, we would have preferred to spend additional time on labelling more data for the sentiment model.
Thus, it could have been possible to fine-tune this model as well.
With our current approach, we only keep the AI future predictions if the sentiment model makes a prediction with a certainty of more than 70\%.
This results in the loss of a few additional statements that we would have available for analysis.
\\
Unfortunately, the location containing the corresponding date on the website does not contain the corresponding date is not consistent.
Accordingly, we would have needed more time for the date extraction.
Providing a year for each statement could illustrate how the perception of a certain topic in the field of AI has changed over time.
Having insights about such trends, allows monitoring the developments in cultural perceptions over time periods.
\\





