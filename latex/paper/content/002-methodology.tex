\section{Methodology}

For the realization of our concept, the following 3 objectives have to be accomplished:

\begin{enumerate}
    \item Obtaining of a sufficiently large data set with different expressions about the topic of AI.
    \item Raw data transformation into a data set that to the target schema illustrated in Table~\ref{data-schema}
    \item Creation of a visualization from which society's perceptions on different topics of AI can be extracted.
\end{enumerate}%
%
%
\textbf{Raw Data Extraction:}
Since a web archive with the corresponding data extraction pipeline is at our disposal, we utilize this one.
The data set consists of long texts.
For that reason the text must be splitted into separate sentences.
Then, sentences about AI can be ex tracted for later processing by applying Regex.
\\
\\
%
\textbf{Data Transformation:}
For later analysis, the data must be converted to required target schema, illustrated in Table~\ref{data-schema}, several challenges have to be handled within this stage.
\\
First, statements about the future must be extracted from all the expressions.
For that purpose we train a model with the ability to distinguish between statements about the future and all other types of terms.
Two further models must be applied to add a sentiment and a topic to every future statement.
The urls of the pages are also included in the data schema for a later consideration.
\\
Now the corresponding data set is prepared for further analysis.
\\
\\
\textbf{Analysis:}
For the analysis we start with a graphical visualization.
Therefore, we decided to group all statements according to their topics.
This way for every topic a sentiment analysis can be conducted.


\begin{table}
\setlength\tabcolsep{2pt} % let LaTeX compute intercolumn whitespace
\footnotesize\centering
\captionsetup{size=footnotesize}
\resizebox{\columnwidth}{!}{%
%\begin{tabular*}{\columnwidth}{@{\extracolsep{\fill}}%
%%%%%%%%%%%%%%%%%%%%%%%%%%%%%%%%%%%%%%%%%%%%%%%%%%%%%%%%%%%%%%%%%%%%%%%%%%%%%%%%%%%%%%%%%%%%%%%%%%%%%%%%%%%%%%%%%%%%%%%%%%%%%%%%%%%%%%%%%%%%%%%%%%%%%%
\begin{tabular}{
    cccc}

\hline

\textbf{statement} & \textbf{Sentiment} & \textbf{Topic} & \textbf{url} \\
\hline
AI can be a risk for many workers. & NEG & finances & ...\\
AI will definitely revolutionize games! & POS &  gaming & ...\\
... & ... & ... & ... \\
\hline
\end{tabular}}
%%%%%%%%%%%%%%%%%%%%%%%%%%%%%%%%%%%%%%%%%%%%%%%%%%%%%%%%%%%%%%%%%%%%%%%%%%%%%%%%%%%%%%%%%%%%%%%%%%%%%%%%%%%%%%%%%%%%%%%%%%%%%%%%%%%%%%%%%%%%%%%%%%%%%%
\caption{\label{data-schema}
Data schema for visualization and analysis
}
\end{table}
