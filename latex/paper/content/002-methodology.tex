\section{Methodology}

For the realization of our concept, the following 3 objectives have to be accomplished:

\begin{enumerate}
    \item Obtaining of a sufficiently large data set with different expressions about the topic of AI.
    \item Transformation of this raw data into a data set that includes only statements about the future, with a topic and a sentiment for each record.
    \item Creation of a visualization from which society's perceptions on different topics of AI can be extracted.
\end{enumerate}%
%
%
For the purpose of extracting sufficiently large data sets about AI, an appropriate sized source is required.
Since a web archive with the corresponding data extraction pipeline is at our disposal, we utilize this one.
The data set consists of long texts.
For that reason the text must be splitted into separate sentences.
Then, regex can be applied to extract sentences about AI for later processing.
\\
\\
After the raw data extraction, a transformation of this data to an appropriate form for visualization and analysis is required.
(Table~\ref{data-schema}) illustrates such a target schema.
To achieve this, several challenges have to be handled within this stage.
\\
First, statements about the future must be extracted from all the expressions.
For that purpose we train a model with the ability to distinguish between statements about the future and all other types of terms.
To add a sentiment to each record, we apply an existing pre-trained model.
Finally, a topic can be associated with each statement.
Once again, we use a pre-trained model.
The urls of the pages are also included in the data schema for a later close validation.
\\
Now the corresponding data set is prepared for further analysis.
\\
\\
For the analysis we start with a graphical visualization.
Therefore, we decided to group all statements according to their topics.
For each topic, a chart can be generated indicating which proportion of the corresponding statements is negative, neutral or positive.

\begin{table}
\setlength\tabcolsep{2pt} % let LaTeX compute intercolumn whitespace
\footnotesize\centering
\captionsetup{size=footnotesize}
\resizebox{\columnwidth}{!}{%
%\begin{tabular*}{\columnwidth}{@{\extracolsep{\fill}}%
%%%%%%%%%%%%%%%%%%%%%%%%%%%%%%%%%%%%%%%%%%%%%%%%%%%%%%%%%%%%%%%%%%%%%%%%%%%%%%%%%%%%%%%%%%%%%%%%%%%%%%%%%%%%%%%%%%%%%%%%%%%%%%%%%%%%%%%%%%%%%%%%%%%%%%
\begin{tabular}{
    cccc}

\hline

\textbf{statement} & \textbf{Sentiment} & \textbf{Topic} & \textbf{url} \\
\hline
AI can be a risk for many workers. & NEG & finances & ...\\
AI will definitely revolutionize games! & POS &  gaming & ...\\
... & ... & ... & ... \\
\hline
\end{tabular}}
%%%%%%%%%%%%%%%%%%%%%%%%%%%%%%%%%%%%%%%%%%%%%%%%%%%%%%%%%%%%%%%%%%%%%%%%%%%%%%%%%%%%%%%%%%%%%%%%%%%%%%%%%%%%%%%%%%%%%%%%%%%%%%%%%%%%%%%%%%%%%%%%%%%%%%
\caption{\label{data-schema}
Data schema for visualization and analysis
}
\end{table}
