\section{Conclusion}
While other groups have previously assessed the perception of AI technologies of certain groups of people, such as investors \citep{manrai2022investor}, librarians \citep{hervieux2021perceptions}, and healthcare professionals \citep{castagno2020perceptions}, or assessed perceptions at current \citep{sankaran2021exploring}, we present the first study to our knowledge that infers people’s feelings towards different areas of AI in the future from online sources. 
The public perception of this technology plays an important part in its development for it influences R\&D funding, adoption, and regulation. \\
Our findings show a slightly positive perception of the future of AI by the English-speaking online community of the past 10 years. Topic-wise, web content seems to deal primarily with areas of application of AI, which, together with the near-neutral average sentiment may suggest a rather matter-of-facts approach in which people have been thinking and writing about AI’s future.
Our approach consists of a pipeline to filter statements about the future and label them with their appropriate sentiments and topics.
In concert, our method offers researchers a way of analyzing statements for opinions and topics in regards to the future for not only artificial intelligence but any subject.