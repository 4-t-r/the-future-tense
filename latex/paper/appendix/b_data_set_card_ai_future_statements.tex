\section{Data Set Card: AI Future Statements}
\\
\\
\textbf{Data Set Description}
\\
\\
The corresponding data set to this Data Set Card originates from \url{https://github.com/atr2384/the-future-tense/blob/develop/stage_2_2_model_pipeline/output/future_statements.csv}.
This english language data set contains 15,541 tuples.
Tuples consist of four attributes (statement (about the future), sentiment, topic, url).
Any information in this data set is a results of combined data collection pipeline processes.
This includes data extraction, filtering, preprocessing and application of several NLP models.
\\
\\
\textbf{Dat Set Motivation}
\\
\\
The overall purpose of this dataset is to provide information on statements about the future that can be analyzed in further steps.
The dataset was created by students from the University of Leipzig in the Big Data and Language Technologies Module of the Webis Group \url{https://huggingface.co/webis}.
\\
\\
\textbf{Data Set Composition and Annotation}
\\
\\
The instances are represented by single-sentence statements which are tempted to be about the future of AI.
Sentiments attribute represents the general sentiment in the corresponding statement. This can be NEG (NEGATIVE), NEU (NEUTRAL), POS (POSITIVE).
A corresponding topic can be of a list of 9 topics including: machine human interface, finance, social media, search engine, computer vision, natural language technologiy, gaming, transhumanism, research computing.
URL contains the URL from which the final AI future statement was extracted.
The dataset consists of 15,541 tuples in total.
\\
\\
\textbf{Noise, Biases and Redundancies}
\\
\\
The main goal of the data collection process was to find future statements about AI topics and labeling it with a sentiment and topic.
The thematic content within the statements can be redundant and some topics are be much more present than others.
Through the fact that the data was collected from WARC-files, actuality and public availability of the url of a statement cannot be guaranteed.
Data in the 'statement'-column is publicly available and does not contain confidential information.
The statements within the dataset do not correspond to the opitions of the project team and is not associated with the authors of this project since the statements were extracted from WARC-files.
\\
\\
\textbf{Data Set Collection Process}
\\
\\
The instances are represented by single-/ or multi-sentence statements which were collected with a WARC-DL data extraction pipeline.
A version of the WARC-DL pipeline which was fitted for the needs of the performed tasks can be found here: \url{https://github.com/atr2384/WARC-DL}.
Since the AI statements are extracted from website HTMLs located inside WARC files, we added the url of each statement in the tuples.
Processed data of the statement attribute is partially observable on the url of each tuple.
Sentiments for the statements were generated by using the SentimentAnalyzer of the open-source library pysentimiento, which was further trained on about 40,000 tweets.
It uses the BERTweet as a base model, pre-trained on english tweets.
Topics were created in a topic model. This topic model consists of a topic discovery step.
This includes a clustering of subcategories through an LDA model and a topic proposition step using GPT-3.
This step also includes manual proposition.
The subsequent assignments of topics to statements was performed by the bart-large-mnli model from Facebook.
The complete workflow for the composition of the data set can be found at \url{https://github.com/atr2384/the-future-tense}
The data in this data set was collected programmatically.
Participants of the extraction process asre graduate students D. Baradari \url{https://huggingface.co/Dunya}, F. Bartels \url{https://huggingface.co/fidsinn}, A. Dewald, J. Peters \url{https://huggingface.co/jpeters92} as part of a data science module of the University of Leipzig.
The data was obtained in 08/2022 but the content of the dataset is independent of the data collection period and can be from earlier periods.
\\
\\
\textbf{Dataset Maintenance}
\\
\\
It is not planned to update the dataset for further work or investigations.