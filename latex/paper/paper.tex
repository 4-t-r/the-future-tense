% This must be in the first 5 lines to tell arXiv to use pdfLaTeX, which is strongly recommended.
\pdfoutput=1
% In particular, the hyperref package requires pdfLaTeX in order to break URLs across lines.

\documentclass[11pt]{article}

% Remove the "review" option to generate the final version.
\usepackage{acl}

% Standard package includes
\usepackage{times}
\usepackage{latexsym}

% Custom package includes
\usepackage{easyReview}

% For proper rendering and hyphenation of words containing Latin characters (including in bib files)
\usepackage[T1]{fontenc}
% For Vietnamese characters
% \usepackage[T5]{fontenc}
% See https://www.latex-project.org/help/documentation/encguide.pdf for other character sets

% This assumes your files are encoded as UTF8
\usepackage[utf8]{inputenc}

% This is not strictly necessary, and may be commented out,
% but it will improve the layout of the manuscript,
% and will typically save some space.
\usepackage{microtype}

\usepackage{graphicx}


\usepackage{titlesec}
\usepackage{hyperref}

\titleclass{\subsubsubsection}{straight}[\subsection]

\newcounter{subsubsubsection}[subsubsection]
\renewcommand\thesubsubsubsection{\thesubsubsection.\arabic{subsubsubsection}}
\renewcommand\theparagraph{\thesubsubsubsection.\arabic{paragraph}} % optional; useful if paragraphs are to be numbered

\titleformat{\subsubsubsection}
  {\normalfont\normalsize\bfseries}{\thesubsubsubsection}{0.5em}{}
\titlespacing*{\subsubsubsection}
{0pt}{1.25ex plus 1ex minus .2ex}{0.5ex plus .2ex}

\makeatletter
\renewcommand\paragraph{\@startsection{paragraph}{5}{\z@}%
  {3.25ex \@plus1ex \@minus.2ex}%
  {-1em}%
  {\normalfont\normalsize\bfseries}}
\renewcommand\subparagraph{\@startsection{subparagraph}{6}{\parindent}%
  {3.25ex \@plus1ex \@minus .2ex}%
  {-1em}%
  {\normalfont\normalsize\bfseries}}
\def\toclevel@subsubsubsection{4}
\def\toclevel@paragraph{5}
\def\toclevel@paragraph{6}
\def\l@subsubsubsection{\@dottedtocline{4}{7em}{4em}}
\def\l@paragraph{\@dottedtocline{5}{10em}{5em}}
\def\l@subparagraph{\@dottedtocline{6}{14em}{6em}}
\makeatother

\setcounter{secnumdepth}{4}
\setcounter{tocdepth}{4}

% If the title and author information does not fit in the area allocated, uncomment the following
%
%\setlength\titlebox{<dim>}
%
% and set <dim> to something 5cm or larger.

\title{The Future Tense - Paper}

% Author information can be set in various styles:
% For several authors from the same institution:
% \author{Author 1 \and ... \and Author n \\
%         Address line \\ ... \\ Address line}
% if the names do not fit well on one line use
%         Author 1 \\ {\bf Author 2} \\ ... \\ {\bf Author n} \\
% For authors from different institutions:
% \author{Author 1 \\ Address line \\  ... \\ Address line
%         \And  ... \And
%         Author n \\ Address line \\ ... \\ Address line}
% To start a seperate ``row'' of authors use \AND, as in
% \author{Author 1 \\ Address line \\  ... \\ Address line
%         \AND
%         Author 2 \\ Address line \\ ... \\ Address line \And
%         Author 3 \\ Address line \\ ... \\ Address line}

\author{Dünya Baradari\\\And
  Finn Bartels \\\And
  Artur Dewald \\\And
  Julia Peters}

\begin{document}
\maketitle
\begin{abstract}
This document is a supplement to the general instructions for *ACL authors.
It contains instructions for using the \LaTeX{} style files for ACL conferences.
The document itself conforms to its own specifications, and is therefore an example of what your manuscript should look like.
These instructions should be used both for papers submitted for review and for final versions of accepted papers.
\end{abstract}

\section{Introduction}

Human-like artificial intelligence (AI) has been exciting and frightening humanity since the antiquity.
Often intertwined with the concept of an artificial man, humanoid automata with the supposed capacity to answer questions and feel emotions have been present among all civilizations, including the ancient Egyptians and Greek \citep{Newquist1994}, Chinese \citep{cohen1986} and Mesopotamians \citep{unat2008}.
Yet, it has been in the past decades that the rise of computing power according to Moore’s Law\footnote{\url{https://www.britannica.com/technology/Moores-law}} has enabled a wide-scale application of AI technologies.
At the time of writing, use cases range from self-driving cars, personalization of ads in online browsing to highly complex predication tasks for protein folding \citep{jumper2021}.
\\
This rapid development of \emph{intelligent machines} in everyday life and applications has led to both hopes and fears among the general population.
\citet{cave2019} identify four dichotomous categories of excitement and fears about artificial intelligence.
These are immortality and inhumanity, ease and obsolescence, gratification and alienation and dominance and uprising (\autoref{dichotomy-categories}).
They further argue that such perceptions, which may not align with reality, can yet influence the development, regulation, and application of AI.
The encouragement of research into AI ethics by various public policy groups and governments may be a reflection of this point \citep{leslie2019}.
\\
In our work, we seek to follow up on the analysis of \citet{cave2019} and examine the views of the English-speaking online community regarding the future of artificial intelligence. We discern the most common clusters of topics that are formed around AI and the average sentiment for each topic using machine learning. To that end, we employ natural language processing (NLP) to extract and analyze statements about the future of AI from the Web Archive \citep{Deckers2022}, a collection of website snapshots which offers us data from the past \textasciitilde10 years.
We apply a pipeline integrating three models on this data.
The first model is a finetuned future model, which is able to recognize statements about the future.
This data is then fed into an existent sentiment classifier to add sentiments.
Finally, our last model assigns a topic to each sentence.
This dataset is subsequently analyzed for its the individual topic clusters.
While our analysis solely concerns artificial intelligence, our pipeline and models offer a way to study online views concerning the future for any topic. By examining the prevalence and sentiment of AI topics specifically, we hope to inform social science researchers, philosophers, and policy makers about the development of artificial intelligence in the general population’s perception, to direct efforts towards a better future with AI.

%%%%%%%%%%%%%%%%%%%%%%%%%%%%%%%%%%%%%%%%%%%%%%%%%%%%%%%%%%%%%%%%%%%%%%%%%%%%%%%%%%%%%%%%%%%%%%%%%%%%%%%%%%%%%%%%%%%%%%%%%%%%%%%
%%%%%%%%%%%%%%%%%%%%%%%%%%%%%%%%%%%%%%%%%%%%%%%%%%%%%%%%%%%%%%%%%%%%%%%%%%%%%%%%%%%%%%%%%%%%%%%%%%%%%%%%%%%%%%%%%%%%%%%%%%%%%%%
\begin{table}[t]
    \centering
    \resizebox{\columnwidth}{!}{%
    \begin{tabular}{lll}
        \toprule
        \textbf{Dichotomy} & \textbf{Hope} & \textbf{Fear} \\
        \midrule
        Immortality and  & Much longer     & Losing one's \\
        Inhumanity       & lives           & identity \\
        \addlinespace[0.7em]
        Ease and         & Life free       & Becoming \\
        Obsolescence     & of work         & redundant \\
        \addlinespace[0.7em]
        Gratification    & AI can fulfill  & Humans will become \\
        and Alienation   & one's desires   & redundant to each other \\
        \addlinespace[0.7em]
        Dominance        & AI offers power & AI will turn \\
        and Uprising     & over others     & against humans \\
        \bottomrule
    \end{tabular}
    }
\caption{\label{dichotomy-categories}
Categories of dichotomies of hopes and fears towards AI.
Based on \citet{cave2019}.
}
\end{table}
%%%%%%%%%%%%%%%%%%%%%%%%%%%%%%%%%%%%%%%%%%%%%%%%%%%%%%%%%%%%%%%%%%%%%%%%%%%%%%%%%%%%%%%%%%%%%%%%%%%%%%%%%%%%%%%%%%%%%%%%%%%%%%%
%%%%%%%%%%%%%%%%%%%%%%%%%%%%%%%%%%%%%%%%%%%%%%%%%%%%%%%%%%%%%%%%%%%%%%%%%%%%%%%%%%%%%%%%%%%%%%%%%%%%%%%%%%%%%%%%%%%%%%%%%%%%%%%

\section{Methodology}

For the realization of our concept, the following 3 objectives have to be accomplished:

\begin{enumerate}
    \item Obtaining of a sufficiently large data set with different expressions about the topic of AI.
    \item Raw data transformation into a data set that to the target schema illustrated in Table~\ref{data-schema}
    \item Creation of a visualization from which society's perceptions on different topics of AI can be extracted.
\end{enumerate}%
%
%
\textbf{Raw Data Extraction:}
Since a web archive with the corresponding data extraction pipeline is at our disposal, we utilize this one.
The data set consists of long texts.
For that reason the text must be splitted into separate sentences.
Then, sentences about AI can be ex tracted for later processing by applying Regex.
\\
\\
%
\textbf{Data Transformation:}
For later analysis, the data must be converted to required target schema, illustrated in Table~\ref{data-schema}, several challenges have to be handled within this stage.
\\
First, statements about the future must be extracted from all the expressions.
For that purpose we train a model with the ability to distinguish between statements about the future and all other types of terms.
Subsequently two further models are applied to add a sentiment and a topic to every future statement. statement.
The urls of the pages are also included in the data schema for a later consideration.
\\
Now the corresponding data set is prepared for further analysis.
\\
\\
\textbf{Analysis:}
For the analysis we start with a graphical visualization.
Therefore, we decided to group all statements according to their topics.
This way for every topic a sentiment analysis can be conducted separately.


\begin{table}
\setlength\tabcolsep{2pt} % let LaTeX compute intercolumn whitespace
\footnotesize\centering
\captionsetup{size=footnotesize}
\resizebox{\columnwidth}{!}{%
%\begin{tabular*}{\columnwidth}{@{\extracolsep{\fill}}%
%%%%%%%%%%%%%%%%%%%%%%%%%%%%%%%%%%%%%%%%%%%%%%%%%%%%%%%%%%%%%%%%%%%%%%%%%%%%%%%%%%%%%%%%%%%%%%%%%%%%%%%%%%%%%%%%%%%%%%%%%%%%%%%%%%%%%%%%%%%%%%%%%%%%%%
\begin{tabular}{
    cccc}

\hline

\textbf{statement} & \textbf{Sentiment} & \textbf{Topic} & \textbf{url} \\
\hline
AI can be a risk for many workers. & NEG & finances & ...\\
AI will definitely revolutionize games! & POS &  gaming & ...\\
... & ... & ... & ... \\
\hline
\end{tabular}}
%%%%%%%%%%%%%%%%%%%%%%%%%%%%%%%%%%%%%%%%%%%%%%%%%%%%%%%%%%%%%%%%%%%%%%%%%%%%%%%%%%%%%%%%%%%%%%%%%%%%%%%%%%%%%%%%%%%%%%%%%%%%%%%%%%%%%%%%%%%%%%%%%%%%%%
\caption{\label{data-schema}
Data schema for visualization and analysis
}
\end{table}

\section{Implementation}
As previously described, our approach consists of three steps.
Initially, we outline the raw data acquiring containing AI expressions.
Towards this objective, we utilize the WARC-DL pipeline \citep{Deckers2022} to extract the data from the given web archive.
For the preparation of this data for analysis, we developed the Model Pipeline which sequentially applies three models to the AI sentences:
\begin{itemize}
    \item Future statements extraction model
    \item Sentiment assignment model
    \item Topic assignment model
\end{itemize}
The Model Pipeline and all the incorporated models are covered in detail below. This way the final data set is generated with the attributes future statement, sentiment and topic.
Finally, the latter created data set serves as a base for a sentiment analysis about the future of AI divided in several subtopics, which we will also address.

\subsection{WARC Data Extraction}
The Webis Group\footnote{\url{https://webis.de}} had provided us with access to 37908 WARC files, with a total amount of \highlight{XXXXXXTB} of website data, and a high-performance computer cluster where we were able to schedule jobs.
Additionally, we decided to utilize the WARC-DL pipeline \citep{Deckers2022}, a Python software pipeline tightly coupled with the WARC endpoint, and using the FastWARC\footnote{\url{https://resiliparse.chatnoir.eu/en/stable/man/fastwarc.html}} library under the hood to iterate over WARC records.
It enabled us to extract text automatically from the WARC records using several customizable filters.
We made some slight modifications to the source code, in order to make it fit our needs better.
\\
Since we were going to analyze AI statements, we did not need all of the text provided by a website html, but grammatically correct statements of the English language.
It is not possible to extract all statements with an accuracy of 100\% using regex only, so we narrowed it down to passages which started with a capital letter and ended with a period or exclamation mark.
We built a regex pattern which extracted a list of all valid statements from a html source.
\\
Furthermore, we were only interested in statements about AI, so we compiled a list of AI keywords beforehand and applied a second regex on the extracted statements.
We compiled a blacklist which contained keywords for filtering out hostnames related to pornographic websites, and a whitelist to only accept top level domains which were most common, and where the websites most likely contained English text.
This was necessary, since initial runs of the WARC-DL pipeline extracted lots of unusable content (including content in languages other than English).
\\
We ran into some problems with the cluster and WARC-DL pipeline, so we could not extract all statements in one job run.
Jobs would suddenly stop extraction because of connection errors or out-of-memory errors, and halt with an Exception.
We separated hostnames in four groups depending on the initial hostname character: \texttt{a-h}, \texttt{i-p}, \texttt{q-x} and \texttt{yz0-9}.
This way we could pin down the problematic websites more precisely.
In the end we managed to work through all WARC files in group \texttt{i-p} and \texttt{yz0-9}, and most of the WARC files in groups \texttt{a-h} and \texttt{q-x}.
The final yield of the WARC data extraction stage was a total amount of 222246 AI statements.
In the next stage, the model pipeline, our objective was to further refine this initial data set.

\subsection{Model Pipeline}
The model pipeline follows the WARC data extraction step and is designed to prepare our final data set, which consists of the future statements and their associated sentiment and topic labels.
The processing within the model pipeline is performed on batches of 30 records each from the WARC-DL output.
\\
First the future model filters out future statements from the corresponding batch.
Subsequently, the chosen sentiment model assigns a sentiment to each future statement.
In this step some future terms can be sorted out.
This concerns the statements to which a sentiment is classified with a probability of less than 70\%.
The remaining future statements receive a topic.
Finally those are persisted in a csv file.
\\
In the following sections \ref{future-model} - \ref{topic-model} we will go into detail about each individual model.
In this context, we describe how the future model was trained and justify our decision for the sentiment and the topic model selection.
Furthermore we outline the choice of our topics and explain, why only those statements are kept which a sentiment with a probability above 70\% can be attributed.

\subsection{Future Model}
\label{future-model}
Since this paper focuses on analyzing statements about the future, a system for distinguishing between future statements and other expressions is required.
In this context, we decided to finetune the DistilBERT \citep{Sanh2019DistilBERTAD} base model that accomplishes this task.
Therefore, in this subsection, the collection of appropriate training data and the subsequent finetuning of the corresponding model is thematized.

\subsubsection{Training Data Set}
\label{training}
In order to provide a suitable data set to establish the future model, we adopted multiple approaches.
At this point, our goal was to compose the data in such a way that we would have a balanced data set with two classes.
The first class should contain future statements and the second all other types of terms.
While two of our group members manually annotated 500 observations each, the other two used an automated mechanism with subsequent verification of the collected data.
\\
One of the automatized approaches involves a web crawler developed on the basis of the python library Beautiful Soup \citep{Richardson2022}.
The text on a page is divided into sentences.
Subsequently every sentence is examined for occurrence of certain terms, as \emph{going to}, \emph{will}, \emph{won't} or \emph{'ll}.
\\
The second automated approach is the sentence extraction tool, which works in several aspects, similar to the web crawler.
At the beginning, it searches the given directory for text files.
If those exist the text is split into sentences and observed for specific expressions, as described above.
\\
To find the phrases that are not future statements, both the web crawler and the sentence extraction tool look only at the corresponding records that do not contain the previously considered expressions.
A careful manual review of all terms gathered by the automated systems was subsequently performed to remove the incorrect records.
\\
Finally, we constructed a data set with 1250 future statements and 1250 other phrases that did not contain future statements.

\subsubsection{Training}
As previously described we used the DistilBERT base model and finetuned it with the data set specified in \ref{training}.
We split the data set of 2500 records into a training and a test set, where the test set contains 20\% of the records.
From the training set we split further 20\% for validation data.
\\
After only two epochs the training ended with an accuracy over 96\% as displayed in (Table~\ref{future-model-train}).
\\
Subsequently we tested the model on our test set containing 500 records never seen by the model and achieved an accuracy of 93.8\%, as seen in the confusion matrix in (Table~\ref{cm}).

%%%%%%%%%%%%%%%%%%%%%%%%%%%%%%%%%%%%%%%%%%%%%%%%%%%%%%%%%%%%%%%%%%%%%%%%%%%%%%%%%%%%%%%%%%%%%%%%%%%%%%%%%%%%%%%%%%%%%%%%%%%%%%%%%%%%%%%%%%%%%%%%%%%%%%
%%%%%%%%%%%%%%%%%%%%%%%%%%%%%%%%%%%%%%%%%%%%%%%%%%%%%%%%%%%%%%%%%%%%%%%%%%%%%%%%%%%%%%%%%%%%%%%%%%%%%%%%%%%%%%%%%%%%%%%%%%%%%%%%%%%%%%%%%%%%%%%%%%%%%%
\begin{table}
    \centering
    \resizebox{\columnwidth}{!}{%
    \begin{tabular}{rrrrr}
        \toprule
        \textbf{Epoch} & \textbf{Train Loss} & \textbf{Train Accuracy} & \textbf{Val. Loss} & \textbf{Val. Accuracy}\\
        \midrule
        0 & 0.3816 & 0.8594 & 0.1547 & 0.9475 \\
        1 & 0.1142 & 0.9613 & 0.1272 & 0.9625 \\
        \bottomrule
    \end{tabular}
    }
\caption{\label{future-model-train}
Training Results
}
\end{table}
%%%%%%%%%%%%%%%%%%%%%%%%%%%%%%%%%%%%%%%%%%%%%%%%%%%%%%%%%%%%%%%%%%%%%%%%%%%%%%%%%%%%%%%%%%%%%%%%%%%%%%%%%%%%%%%%%%%%%%%%%%%%%%%%%%%%%%%%%%%%%%%%%%%%%%
%%%%%%%%%%%%%%%%%%%%%%%%%%%%%%%%%%%%%%%%%%%%%%%%%%%%%%%%%%%%%%%%%%%%%%%%%%%%%%%%%%%%%%%%%%%%%%%%%%%%%%%%%%%%%%%%%%%%%%%%%%%%%%%%%%%%%%%%%%%%%%%%%%%%%%

%%%%%%%%%%%%%%%%%%%%%%%%%%%%%%%%%%%%%%%%%%%%%%%%%%%%%%%%%%%%%%%%%%%%%%%%%%%%%%%%%%%%%%%%%%%%%%%%%%%%%%%%%%%%%%%%%%%%%%%%%%%%%%%%%%%%%%%%%%%%%%%%%%%%%%
%%%%%%%%%%%%%%%%%%%%%%%%%%%%%%%%%%%%%%%%%%%%%%%%%%%%%%%%%%%%%%%%%%%%%%%%%%%%%%%%%%%%%%%%%%%%%%%%%%%%%%%%%%%%%%%%%%%%%%%%%%%%%%%%%%%%%%%%%%%%%%%%%%%%%%
\begin{table}[b]
    \centering
    \tiny
    \resizebox{\columnwidth}{!}{%
    \begin{tabular}{lrr}
        \toprule
        {} & 1 & 2 \\
        \midrule
        True future statements & 232 & 13 \\
        True non-future statements & 18 & 237 \\
        \bottomrule
    \end{tabular}
    }
\caption{\label{cm}
Confusion Matrix\\
column 1: classified as future statements\\
column 2: classified as no future statements
}
\end{table}
%%%%%%%%%%%%%%%%%%%%%%%%%%%%%%%%%%%%%%%%%%%%%%%%%%%%%%%%%%%%%%%%%%%%%%%%%%%%%%%%%%%%%%%%%%%%%%%%%%%%%%%%%%%%%%%%%%%%%%%%%%%%%%%%%%%%%%%%%%%%%%%%%%%%%%
%%%%%%%%%%%%%%%%%%%%%%%%%%%%%%%%%%%%%%%%%%%%%%%%%%%%%%%%%%%%%%%%%%%%%%%%%%%%%%%%%%%%%%%%%%%%%%%%%%%%%%%%%%%%%%%%%%%%%%%%%%%%%%%%%%%%%%%%%%%%%%%%%%%%%%

\subsection{Sentiment Model}
In order to assign sentiments to future statements for later analysis, we decided to select a ready-trained model.
The chosen sentiment model is the SentimentAnalyzer of the open-source library pysentimiento \citep{perez2021pysentimiento}, which was further trained on about 40k tweets.
It uses the BERTweet \citep{bertweet} as a base model, pre-trained on english tweets.

\subsubsection{Evaluation}
To evaluate the SentimentAnalyzer, we annotated 500 future statements, which were previously used for training the future model, as negative, positive or neutral and received an accuracy of about 65\%.
\\
We then analyzed all misclassified statements.
We noticed that some of the statements could not be assigned impartially to one of the three categories.
An Example is \emph{``AI will reinvent how we think about education''}.
In the case of the sentence, we disagreed on whether we should value the sentence as neutral or positive and decided to use the neutral label.
Subsequently, this statement was given a positive rating by the model.
On closer examination of the statements that were labeled differently by us and by the model, we found over 90\% of the labels given by the model to be valid, if these annotations were assigned with probability over 70\%.
For this reason, we decided to keep only statements about the future if the sentiment model assigned an annotation with a confidence above 70\%.

\subsection{Topic Model}
\label{topic-model}

\section{Results}
Looking on the distribution of sentiments, it appears that the majority of statements were commented as neutral statements (69\%). The proportion of positive annotated statements (21\%) is about twice that of negative annotated statements (11\%). This shows that overall there is a slight tendency toward a positive attitude on the future of AI (\autoref{fig:bar_sentiments}). In figure \ref{fig:bar_topics} it can be seen that neutral statements dominate each of the 9 topics. With two exceptions, Gaming and Machine Human Interface, there are visibly more positive than negative statements on each topic. 
\\
\\
When analyzing the topics, we find that the statements are not equally distributed among all topics. While we divided all statements into 9 topics, Machine Human Interface describes about half of all statements (48\%). Gaming as well as Natural Language Technology account for about 15\% of all statements (\autoref{fig:pie_topics_by_occ}). 
\\
\\
In the distribution of subtopics we can see a dominance of some subtopics too. The subtopic Data is associated with 21\% of all statements, as well as Autopilot. Other dominant subtopics are Intelligence (19\%), Recognition (12\%), Computer (8\%) and Supercomputer (7\%) (\autoref{fig:pie_subtopics_by_occ}). 
\\
\\
The average sentiment of all statements is at 0.1. This means a slight tendency to positive sentiment. The average sentiment of most of the 9 topics is majorly neutral. The topics of Transhumanism, Natural Language Technology, and Research Computing have the most positive sentiment on average. The most negative sentiment on average can be seen at Gaming and Search Engine. 3 of the 5 most common subtopics of Gaming have a sentiment score of less than 0 (\autoref{fig:pie_topics_subtopics_by_occ_sent_neu}). 
\section{Discussion}
Since we had a limited time for this project, there are some aspects where we would have liked to continue our work.
From a technical point of view, we would have preferred to spend additional time on labelling more data for the sentiment model.
Thus, it could have been possible to fine-tune this model as well.
With our current approach, we only keep the AI future predictions if the sentiment model makes a prediction with a certainty of more than 70\%. 
This results in the loss of a few additional statements that we would have available for analysis.
\\
Unfortunately, the location containing the corresponding date on the website does not contain the corresponding date is not consistent. 
Accordingly, we would have needed more time for the date extraction.
Providing a year for each statement could illustrate how the perception of a certain topic in the field of AI has changed over time.
Having insights about such trends, allows monitoring the developments in cultural perceptions over time periods.
\input{content/006-conclusion}

% Entries for the entire Anthology, followed by custom entries
\bibliography{anthology,custom}

\appendix
\section{Model Card: Future Statement Model}
This model is a finetuned on 2500 expressions, which contained 1250 future statements. distilbert-base-uncased serves as a base model 
\\
\\
%
\textbf{Model Description}
%
\begin{itemize}
    \item Huggingface name: fidsinn/distilbert-base-future
    \item Creation Date: 11/08/22
    \item Version: 1.0
    \item model type: text classification
\end{itemize}%
%
\textbf{Intended Use \& Limitations}
%
\begin{itemize}
    \item The primary intended use is the classification of input into a future or non-future sentence/statement.
    \item The model is primarily intended to be used by researchers to filter or label a large number of sentences according to the grammatical tense of the input.
\end{itemize}%
%
\textbf{Hyperparameters}
\\
\\
The following hyperparameters were used during training
\begin{itemize}
    \item optimizer: 
    name: Adam, learning\_rate: 5e-05, decay: 0.0, beta\_1: 0.9, beta\_2: 0.999, epsilon: 1e-07, amsgrad: False
    \item training\_precision: float32
\end{itemize}%
\label{sec:appendix}
%
\textbf{Training Results}
\\
%%%%%%%%%%%%%%%%%%%%%%%%%%%%%%%%%%%%%%%%%%%%%%%%%%%%%%%%%%%%%%%%%%%%%%%%%%%%%%%%%%%%%%%%%%%%%%%%%%%%%%%%%%%%%%%%%%%%%%%%%%%%%%%%%%%%%%%%%%%%%%%%%%%%%%
%%%%%%%%%%%%%%%%%%%%%%%%%%%%%%%%%%%%%%%%%%%%%%%%%%%%%%%%%%%%%%%%%%%%%%%%%%%%%%%%%%%%%%%%%%%%%%%%%%%%%%%%%%%%%%%%%%%%%%%%%%%%%%%%%%%%%%%%%%%%%%%%%%%%%%
\begin{table}[ht]
\setlength\tabcolsep{2pt} % let LaTeX compute intercolumn whitespace
\footnotesize\centering
\captionsetup{size=footnotesize}
\resizebox{\columnwidth}{!}{%
%\begin{tabular*}{\columnwidth}{@{\extracolsep{\fill}}%
%%%%%%%%%%%%%%%%%%%%%%%%%%%%%%%%%%%%%%%%%%%%%%%%%%%%%%%%%%%%%%%%%%%%%%%%%%%%%%%%%%%%%%%%%%%%%%%%%%%%%%%%%%%%%%%%%%%%%%%%%%%%%%%%%%%%%%%%%%%%%%%%%%%%%%
\begin{tabular}{
    ccccc}

\hline

\textbf{Epoch} & \textbf{Train Loss} & \textbf{Train Accuracy} & \textbf{Val. Loss} & \textbf{Val. Accuracy}\\
\hline
0 & 0.3816 & 0.8594 & 0.1547 & 0.9475 \\
1 & 0.1142 & 0.9613 & 0.1272 & 0.9625 \\
\hline
\end{tabular}}
%%%%%%%%%%%%%%%%%%%%%%%%%%%%%%%%%%%%%%%%%%%%%%%%%%%%%%%%%%%%%%%%%%%%%%%%%%%%%%%%%%%%%%%%%%%%%%%%%%%%%%%%%%%%%%%%%%%%%%%%%%%%%%%%%%%%%%%%%%%%%%%%%%%%%%
\caption{\label{future-model-train}
Training Results
}
\end{table}
%%%%%%%%%%%%%%%%%%%%%%%%%%%%%%%%%%%%%%%%%%%%%%%%%%%%%%%%%%%%%%%%%%%%%%%%%%%%%%%%%%%%%%%%%%%%%%%%%%%%%%%%%%%%%%%%%%%%%%%%%%%%%%%%%%%%%%%%%%%%%%%%%%%%%%
%%%%%%%%%%%%%%%%%%%%%%%%%%%%%%%%%%%%%%%%%%%%%%%%%%%%%%%%%%%%%%%%%%%%%%%%%%%%%%%%%%%%%%%%%%%%%%%%%%%%%%%%%%%%%%%%%%%%%%%%%%%%%%%%%%%%%%%%%%%%%%%%%%%%%%
%
%
\\
\textbf{Framework versions}
\begin{itemize}
    \item Transformers 4.18.0
    \item Tensorflow 2.8.0
    \item Tokenizers 0.12.1
\end{itemize}%
\textbf{Test Set Results}
%%%%%%%%%%%%%%%%%%%%%%%%%%%%%%%%%%%%%%%%%%%%%%%%%%%%%%%%%%%%%%%%%%%%%%%%%%%%%%%%%%%%%%%%%%%%%%%%%%%%%%%%%%%%%%%%%%%%%%%%%%%%%%%%%%%%%%%%%%%%%%%%%%%%%%
%%%%%%%%%%%%%%%%%%%%%%%%%%%%%%%%%%%%%%%%%%%%%%%%%%%%%%%%%%%%%%%%%%%%%%%%%%%%%%%%%%%%%%%%%%%%%%%%%%%%%%%%%%%%%%%%%%%%%%%%%%%%%%%%%%%%%%%%%%%%%%%%%%%%%%
\begin{table}
\small
\captionsetup{size=footnotesize}
%\begin{tabular*}{\columnwidth}{@{\extracolsep{\fill}}%
%%%%%%%%%%%%%%%%%%%%%%%%%%%%%%%%%%%%%%%%%%%%%%%%%%%%%%%%%%%%%%%%%%%%%%%%%%%%%%%%%%%%%%%%%%%%%%%%%%%%%%%%%%%%%%%%%%%%%%%%%%%%%%%%%%%%%%%%%%%%%%%%%%%%%%
\begin{tabular}{
    c|cc}
\hline
& 1 & 2\\
\hline
true future statement & 253 & 4 \\
true no future statements & 4 & 239 \\
\hline
\end{tabular}
%%%%%%%%%%%%%%%%%%%%%%%%%%%%%%%%%%%%%%%%%%%%%%%%%%%%%%%%%%%%%%%%%%%%%%%%%%%%%%%%%%%%%%%%%%%%%%%%%%%%%%%%%%%%%%%%%%%%%%%%%%%%%%%%%%%%%%%%%%%%%%%%%%%%%%
\caption{\label{cm}
Confusion Matrix\\
column 1: classified as future statements\\
column 2: classified as no future statements
}
\end{table}
%%%%%%%%%%%%%%%%%%%%%%%%%%%%%%%%%%%%%%%%%%%%%%%%%%%%%%%%%%%%%%%%%%%%%%%%%%%%%%%%%%%%%%%%%%%%%%%%%%%%%%%%%%%%%%%%%%%%%%%%%%%%%%%%%%%%%%%%%%%%%%%%%%%%%%
%%%%%%%%%%%%%%%%%%%%%%%%%%%%%%%%%%%%%%%%%%%%%%%%%%%%%%%%%%%%%%%%%%%%%%%%%%%%%%%%%%%%%%%%%%%%%%%%%%%%%%%%%%%%%%%%%%%%%%%%%%%%%%%%%%%%%%%%%%%%%%%%%%%%%%

\end{document}
